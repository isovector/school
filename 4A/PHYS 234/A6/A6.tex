\documentclass[12pt]{article}

\usepackage[margin=1in]{geometry}
\usepackage{amsmath}
\usepackage{amsfonts}
\usepackage{braket, units, enumitem}

\begin{document}

\newcounter{set}
\setcounter{set}{1}
\newcounter{problem}[set]
\newcommand{\problem}{{\vspace{2\baselineskip}\noindent\large \bfseries Problem~\arabic{set}:}\\\refstepcounter{set}}
\newcommand{\problemsub}{\refstepcounter{problem}{\vspace{2\baselineskip}\noindent\large \bfseries Problem~\arabic{set} \roman{problem}:}\\}
\newcommand{\problemasub}{\refstepcounter{problem}{\vspace{2\baselineskip}\noindent\large \bfseries Problem~\arabic{set} \alph{problem}:}\\}
\newcommand*\colvec[3][]{\begin{bmatrix}\ifx\relax#1\relax\else#1\\\fi#2\\#3\end{bmatrix}}
\newcommand{\Ketbra}[1]{\Ket{#1}\Bra{#1}}
\newcommand{\Ketbrad}[2]{\Ket{#1}_{#2}{}_{#2}\Bra{#1}}

\nocite{*}

\title{PHYS 234 - A6}

\author{Alexander Maguire \\ 
amaguire@uwaterloo.ca \\
20396195}
\maketitle




\problemasub
\begin{align*}
\rho &= \sum_i {p_i\Ketbra{\psi_i}} \\
     &= 0.2 \Ketbrad{-}{x} + 0.75 \Ketbrad{+}{y} + 0.05 \Ketbra{+} \\
     &= 0.2 \begin{bmatrix}0.707\\-0.707\end{bmatrix} + 0.75 \begin{bmatrix}0.707\\0.707i\end{bmatrix} + 0.05 \begin{bmatrix}1\\0\end{bmatrix} \\
     &= \begin{bmatrix}0.525&-0.1-0.375i\\-0.1+0.375i&0.475\end{bmatrix}
\end{align*}

\problemasub
\begin{align*}
    P_{x+} &= 0.2 |{}_x\Braket{+|-}_x|^2 + 0.75 |{}_x\Braket{+|+}_y|^2 + 0.05 |{}_x\Braket{+|+}|^2 \\
           &= 0.4 \\ \\
    P_{x-} &= 1 - P_{x+} \\
           &= 0.6
\end{align*}

\problemasub
\begin{align*}
    \Braket{\psi} &= 0.4 \frac{\hbar}{2} - 0.6 \frac{\hbar}{2} \\
                  &= -0.1 \hbar
\end{align*}

\problemasub
\begin{align*}
    \Braket{\mathbf{S_x} \rho} &= Tr \biggl\{ \frac{\hbar}{2} \begin{bmatrix}0&1\\1&0\end{bmatrix} \begin{bmatrix}0.525&-0.1-0.375i\\-0.1+0.375i&0.475\end{bmatrix} \biggr\} \\
                               &= \frac{\hbar}{2} Tr \biggl\{ \begin{bmatrix}-0.1+0.375i&0.475\\0.525&-0.1-0.375i\end{bmatrix} \biggr\} \\
                               &= -0.1 \hbar
\end{align*}

\stepcounter{set}
\problem

\begin{align*}
    P(x+) &= |{}_{x}\Braket{+|+}|^2 \\
          &= 0.5
\end{align*}

Conditional on the observance of $\Ket{+}_x$, we continue:

\begin{align*}
    P(+|x+) &= |\Braket{+|+}_x|^2 \\
            &= 0.5 \\
    P(-|x+) &= |\Braket{-|+}_x|^2 \\
            &= 0.5 \\
    P(+) &= P(x+) \times P(+|x+) = 0.25 \\
    P(-) &= P(x+) \times P(-|x+) = 0.25 \\
\end{align*}

\newpage
\problem
\begin{align*}
    \theta &= \frac{\pi}{4} \\
    \phi &= \frac{5\pi}{3} \\ \\
    \Ket{\psi}_n &= \cos{\frac{\theta}{2}}\Ket{+} + \sin^2{\frac{\theta}{2}}e^{i\phi}\Ket{-} \\
                 &= 0.924\Ket{+} + (0.191 - 0.331i)\Ket{-} \\ \\
    P_{y+} &= |{}_y\Braket{+|\psi}_n|^2 \\
           &= 0.194 \\
    P_{y-} &= 1 - P_{y+} \\
           &= 0.806
\end{align*}


\problemasub
\begin{align*}
    AB - BA &= \begin{bmatrix}
    a_1 b_1 & 0 & 0 \\
    0 & 0 & a_2 b_2 \\
    0 & a_3 b_2 & 0
    \end{bmatrix} -
    \begin{bmatrix}
        a_1 b_1 & 0 & 0 \\
        0 & 0 & a_3 b_2 \\
        0 & a_2 b_2 & 0
    \end{bmatrix} \\
    &\ne \hat{0}
\end{align*}

No, these operators do not commute with one another.

\newpage
\problemasub
Because A is a diagonal basis, it must be expressed in its own basis. Therefore:
\begin{align*}
    A_1 &= a_1 \\
    A_2 &= a_2 \\
    A_3 &= a_3 \\
    \\
    \Ket{a_1} &= \begin{bmatrix}1\\0\\0\end{bmatrix} \\
    \Ket{a_2} &= \begin{bmatrix}0\\1\\0\end{bmatrix} \\
    \Ket{a_3} &= \begin{bmatrix}0\\0\\1\end{bmatrix}
\end{align*}

Unfortunately B is not so nice, we must compute its eigenvalues:

\begin{align*}
    \det{B - \lambda I} = 0 \\
    \det{
    \begin{bmatrix}
        b_1 -\lambda & 0 & 0 \\
        0 & -\lambda & b_2 \\
        0 & b_2 & -\lambda
    \end{bmatrix}} &= 0 \\
    (b_1 - \lambda)(\lambda^2 - b_2^2) = 0 \\
    \lambda &= \{ b_1, b_2, -b_2 \} \\
\end{align*}\begin{align*}
    \begin{bmatrix}
        b_1 & 0 & 0 \\
        0 & 0 & b_2 \\
        0 & b_2 & 0
    \end{bmatrix}
    \begin{bmatrix}a\\b\\c\end{bmatrix} 
        &= b_1\begin{bmatrix}a\\b\\c\end{bmatrix} \\
    \Ket{b_1} &= \begin{bmatrix}1\\0\\0\end{bmatrix} \\
\end{align*}\begin{align*}
    \begin{bmatrix}
        b_1 & 0 & 0 \\
        0 & 0 & b_2 \\
        0 & b_2 & 0
    \end{bmatrix}
    \begin{bmatrix}a\\b\\c\end{bmatrix} 
        &= b_2\begin{bmatrix}a\\b\\c\end{bmatrix} \\
    \Ket{b_2} &= \frac{1}{\sqrt{2}}\begin{bmatrix}0\\1\\1\end{bmatrix} \\
\end{align*}\begin{align*}
    \begin{bmatrix}
        b_1 & 0 & 0 \\
        0 & 0 & b_2 \\
        0 & b_2 & 0
    \end{bmatrix}
    \begin{bmatrix}a\\b\\c\end{bmatrix} 
        &= -b_2\begin{bmatrix}a\\b\\c\end{bmatrix} \\
    \Ket{-b_2} &= \frac{1}{\sqrt{2}}\begin{bmatrix}0\\-1\\1\end{bmatrix}
\end{align*}

\problemasub
\begin{align*}
    P_{b_1} &= |\Braket{b_1|2}|^2 \\
            &= \biggl|\begin{bmatrix}1&0&0\end{bmatrix}\begin{bmatrix}0\\1\\0\end{bmatrix}\biggr|^2 \\
            &= 0 \\ \\
    P_{b_2} &= |\Braket{b_2|2}|^2 \\
            &= \biggl|\begin{bmatrix}0&0.707&0.707\end{bmatrix}\begin{bmatrix}0\\1\\0\end{bmatrix}\biggr|^2 \\
            &= 0.5 \\ \\
    P_{-b_2} &= |\Braket{-b_2|2}|^2 \\
            &= \biggl|\begin{bmatrix}0&-0.707&0.707\end{bmatrix}\begin{bmatrix}0\\1\\0\end{bmatrix}\biggr|^2 \\
            &= 0.5
\end{align*}\begin{align*}
    P_{a_1|b_2} &= \biggl|\begin{bmatrix}1&0&0\end{bmatrix}\begin{bmatrix}0\\0.707\\0.707\end{bmatrix}\biggr|^2 \\
                &= 0 \\ \\
    P_{a_2|b_2} &= \biggl|\begin{bmatrix}0&1&0\end{bmatrix}\begin{bmatrix}0\\0.707\\0.707\end{bmatrix}\biggr|^2 \\
                &= 0.5 \\ \\
    P_{a_3|b_2} &= \biggl|\begin{bmatrix}0&0&1\end{bmatrix}\begin{bmatrix}0\\0.707\\0.707\end{bmatrix}\biggr|^2 \\
                &= 0.5
\end{align*}\begin{align*}
    P_{a_1|-b_2} &= \biggl|\begin{bmatrix}1&0&0\end{bmatrix}\begin{bmatrix}0\\-0.707\\0.707\end{bmatrix}\biggr|^2 \\
                &= 0 \\ \\
    P_{a_2|-b_2} &= \biggl|\begin{bmatrix}0&1&0\end{bmatrix}\begin{bmatrix}0\\-0.707\\0.707\end{bmatrix}\biggr|^2 \\
                &= 0.5 \\ \\
    P_{a_3|-b_2} &= \biggl|\begin{bmatrix}0&0&1\end{bmatrix}\begin{bmatrix}0\\-0.707\\0.707\end{bmatrix}\biggr|^2 \\
                &= 0.5
\end{align*}


\problemasub
(a) and (c) are related in that the answers for (c) would be the same if operators A and B commuted.

\end{document}
