\documentclass[12pt]{article}

\usepackage[margin=1in]{geometry}  % set the margins to 1in on all sides
\usepackage{amsmath}               % great math stuff
\usepackage{amsfonts}              % for blackboard bold, etc
\usepackage{braket, units, enumitem}

\begin{document}


\newcounter{set}
\setcounter{set}{1}
\newcounter{problem}[set]
\newcommand{\problem}{{\vspace{2\baselineskip}\noindent\large \bfseries Problem~\arabic{set}:}\\\refstepcounter{set}}
\newcommand{\problemsub}{\refstepcounter{problem}{\vspace{2\baselineskip}\noindent\large \bfseries Problem~\arabic{set}\roman{problem}:}\\}
\newcommand{\problemasub}{\refstepcounter{problem}{\vspace{2\baselineskip}\noindent\large \bfseries Problem~\arabic{set}\alph{problem}:}\\}
\newcommand*\colvec[3][]{\begin{pmatrix}\ifx\relax#1\relax\else#1\\\fi#2\\#3\end{pmatrix}}



\nocite{*}

\title{PHYS 234 - A2}

\author{Alexander Maguire \\ 
amaguire@uwaterloo.ca \\
20396195}

\maketitle

\problem %1

Historically, the relevant length scale of the Davisson-Gerner experiment was approximately $\unit[1]{\AA}$, but is closer to $\unit[1\times10^{-4}][\AA]$ with modern breakthroughs in aperture design.

It is so hard to detect wave-like phenomena for macroscopic objects because by the de Broglie wavelength equation, wavelength is inversely proportional to momentum. Macroscopic objects have a large mass, which implies they have a large momentum, which means their wavelengths are far too large to be measured by the Davisson-Gerner experimental apparatus. 


\problem %2
\begin{enumerate}[label=\alph{*})]
\item Since additive interference happens when $2d\sin{\theta} = n\lambda$ for $n \in \mathbb{Z}$, in order to maximize our wavelength $\theta = \pi/2$ when measured from the plane of refraction. For a first-order maximum, $n = 1$.
is
We are looking at a diagonal plane of atoms, each $\unit[0.91]{\AA}$ from one another measured on the axis. Because $d$ measures the distance between the planes, $d = \unit[0.91]{\AA}/\sqrt{2} = \unit[0.64]{\AA}$.

$$
\lambda =  2d\sin{\pi/2} = \unit[1.3]{\AA}
$$

\newpage
\item Assuming our electrons are traveling at non-relativistic speeds, $K = \frac{1}{2}mv^2 = \unit[300]{eV}$. Solving this gives $v = \unit[1.03\times10^7]{m/s}$, or about $0.0343c$, justifying our use of the classical equation.

\begin{align*}
\lambda &= \frac{h}{p} = \frac{h}{m_ev} = \unit[0.71]{\AA} \\
\\
\theta &= \arcsin{\frac{\lambda}{2d}} = \unit[0.59]{rad}
\end{align*}
\end{enumerate}
 
\problem %3
\begin{enumerate}[label=\alph{*})]
\item 
$$\mu = \frac{m_1m_2}{m_1+m_2} = \unit[0.5]{g}$$
$$k = \mu(2\pi\nu)^2 = \unit[197]{kg/s^2}$$
\begin{align*}
U = \frac{1}{2}kx^2 &= \unit[9.9\times 10^{-3}]{J} \\ 
&= \unit[6.20\times10^{16}]{eV}
\end{align*}
$$n = \frac{U}{h\nu} = 1.49\times10^{29}$$

\item $$\frac{E_{1.49\times10^{29}+1}}{E_{1.49\times10^{29}}} - 1 \approx 0.000\%$$
No change is detectable whatsoever.

\item $$E_1 = h\nu= \unit[4.14\times10^{-1}]{eV}$$
$$E_2 = \unit[0.827]{eV}$$
$$E_3 = \unit[1.24]{eV}$$

\newpage
\item
$$\mu = \frac{m_1m_2}{m_1+m_2} = \unit[8.5\times10^{-25}]{g}$$
$$k = \mu(2\pi\nu)^2 = \unit[336]{kg/s^2}$$

\begin{align*}
x_1 = \sqrt{2 \frac{E_1}{k}} &= \unit[0.21]{\AA} \\
x_2 = \sqrt{2 \frac{E_2}{k}} &= \unit[0.29]{\AA} \\
x_3 = \sqrt{2 \frac{E_3}{k}} &= \unit[0.36]{\AA} \\
\end{align*}
\end{enumerate}


\problemsub %4i
\begin{align*}
\text{let $\hat{A}$} &= \left(\begin{matrix}
  2 & 7 \\
  7 & 2 \\
 \end{matrix}\right) \\
0 &= \det{\hat{A} - \lambda\hat{I}} \\
 &= (2-\lambda)^2-7^2 \\
 &= 4-4\lambda+\lambda^2-49 \\
 &= (\lambda - 9)(\lambda + 5)
\end{align*}
\begin{align*}
\hat{A}\colvec{a}{b} &= \lambda_1\colvec{a}{b} \\
\colvec{2a + 7b}{7a + 2b} &= 9\colvec{a}{b} \\
v_1 &= \colvec{1}{1}
\end{align*}
\begin{align*}
\hat{A}\colvec{a}{b} &= \lambda_2\colvec{a}{b} \\
\colvec{2a + 7b}{7a + 2b} &= -5\colvec{a}{b} \\
v_2 &= \colvec{-1}{1}
\end{align*}



\problemsub %4ii
\begin{align*}
\text{let $\hat{B}$} &= \left(\begin{matrix}
  1 & -2 \\
  1 & 3 \\
 \end{matrix}\right) \\
0 &= \det{\hat{B} - \lambda\hat{I}} \\
 &= (1-\lambda)(3-\lambda) +2 \\
 &= (\lambda - (2-i))(\lambda + (2+i))
\end{align*}
\begin{align*}
\hat{B}\colvec{a}{b} &= \lambda_1\colvec{a}{b} \\
\colvec{a - 2b}{a + 3b} &= (2-i)\colvec{a}{b} \\
v_1 &= \colvec{-1-i}{1}
\end{align*}
\begin{align*}
\hat{B}\colvec{a}{b} &= \lambda_2\colvec{a}{b} \\
\colvec{a - 2b}{a + 3b} &= (2+i)\colvec{a}{b} \\
v_2 &= \colvec{-1+i}{1}
\end{align*}

\problemsub %4iii
\begin{align*}
\text{let $k$} &= \frac{\hbar}{2} \\
\text{let $\hat{C}$} &= \left(\begin{matrix}
  0 & -i \\
  i & 0 \\
 \end{matrix}\right) \\
0 &= \det{\hat{C} - \lambda\hat{I}} \\
 &= \lambda^2-1 \\
 &= (k\lambda - 1)(k\lambda + 1)
\end{align*}
\begin{align*}
k\hat{C}\colvec{a}{b} &= \lambda_1\colvec{a}{b} \\
k\colvec{-a-ib}{ia-b} &= \colvec{0}{0} \\
v_1 &= k\colvec{-i}{1}
\end{align*}
\begin{align*}
k\hat{C}\colvec{a}{b} &= \lambda_2\colvec{a}{b} \\
k\colvec{a-ib}{ia+b} &= \colvec{0}{0} \\
v_2 &= k\colvec{i}{1}
\end{align*}

\stepcounter{set}
\problem %5

$$\hat{A} = \left(\begin{matrix}
-1 & 1 & i \\
2 & 0 & 3 \\
2i & -2i & 2
\end{matrix}\right) \qquad
\hat{B} = \left(\begin{matrix}
2 & 0 & -i \\
0 & 1 & 0 \\
i & 3 & 2
\end{matrix}\right)$$

\begin{enumerate}[label=\alph{*})]
\item %a
$$\hat{A}+\hat{B} = \left(\begin{matrix}
1 & 1 & 0 \\
2 & 1 & 3 \\
3i & 3-2i & 4
\end{matrix}\right)$$

\item %b
$$\hat{A}\hat{B} = \left(\begin{matrix}
-3 & 1+3 i & 3 i \\
4+3 i & 9 & 6-2 i \\
6 i & 6-2 i & 6 \\
\end{matrix}\right)$$

\item %c
$$\left[\hat{A}, \hat{B}\right] = 
\left(\begin{matrix}
-1 & 1 & i & 2 & 0 & -i\\
2 & 0 & 3 & 0 & 1 & 0 \\
2i & -2i & 2 & i & 3 & 2
\end{matrix}\right)$$

\item %d
$$\bar{\hat{A}} = \left(\begin{matrix}
-1 & 2 & 2i \\ 
1 & 0 & -21 \\
i  & 3 & 2
\end{matrix}\right)$$

\item %e
$$\hat{A}^* = \left(\begin{matrix}
-1 & 1 & -i \\
2 & 0 & 3 \\
-2i & 2i & 2
\end{matrix}\right)$$

\item %f
$$\hat{A}^\dagger = \left(\begin{matrix}
-1 & 1 & i \\
2 & 0 & 3 \\
2i & -2i & 2
\end{matrix}\right)$$

\item %g
$Tr(\hat{B}) = 2 + 1 + 2 = 5$
\end{enumerate}


\end{document}