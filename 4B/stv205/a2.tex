\documentclass{article}

\usepackage{color,soul,todonotes,hyperref}

\title{Anonymous Taken out of Context}
\author{Alexander Maguire, \#20396195}

\begin{document}

\maketitle

There is no problem that Anonymous is trying to address. To even ask such a question implies one has been bamboozled.
Anonymous as an activist group is but a fa\c{c}ade for Anonymous as a loosely associated group of socially-awkward
hoodlums. As a group, they have no motives, no incentives, and no goals. As scathing a remark as this sounds, it is not
without support. This essay aims to provide more insight into the psyches of those involved, and what such a viewpoint
can say in aggregate about the group.

Indeed, it is a great folly to assume that Anonymous is any single one agent, or even that it has a coherent set of
values. Perhaps one of the most glaring flaws in all of human psychology is its implicit assumption of identity once a
concept has been given a label. While this eliding might work in most everyday contexts, in the case of Anonymous we
find our naive approaches to reasoning lacking. The ``entity'' we describe as Anonymous is not an entity at all so much
as an ever-shifting unrelated group of people whose only uniting factor is that they now have a veil behind which to
hide their adolescent chicanery.

Rooting out Anonymous' ultimate goal is ultimately an intractable problem, namely because they don't have one. The
majority of the tropes in their memeplex come from an online image-board called 4chan, a website known primarily for
eschewing any form of identity whatsoever. The memetics of modern day ``hacktivist'' Anonymous are certainly a spiritual
successor, if not one-and-the-same, to the denizens of 4chan. The anonymous members of 4chan have been responsible for
many hijinx on the internet over the years, perhaps most notably gaming the TIME 100 poll in 2009, where they
successfully manipulated the results into an obscure acrostic. It is important to note that such an event did not
further any political agenda, its participants merely found it to be fun, and cleaved them closer together as a
community.

A curtain of anonymity is convenient for mischievous types; when taken to the extreme, it ensures that not even those
involved know the identities of their co-conspirators. Such a decentralized organizational structure implies the
conceptual group can outlive any of its original founding members -- after all, how would anyone ever know? As such, any
potential prankster now has the ability to attribute their atrocities to Anonymous. The author would like to add, hoping
the reader will give him the benefit of not intending to sound ad-hominem, that there is likely a social (as in
``social-life'') component whose influence here cannot be understated.

Having spent a few months of his life frequenting the image-board 4chan, the author can speak from experience that its
denizens are not those whom one might describe as being ``well-adjusted''. It is the social misfits, the outcasts from
everyday society, who congregate online anonymously. Their posing under the mantle of activism begins now to make some
sense; their projected images are no longer of weirdos, but of heroes -- those who protect and defend democracy and
free-speech. This should not be interpreted for anything other than what it is, however: propaganda. Mischief is
undoubtedly the primary goal, with social credit being the proverbial cherry on top.

It is for these reasons I would answer all of the conclusive questions in this paper as ``yes'': these tactics are an
appropriate solution, DDoS \textit{can} be considered a legitimate form of civil disobedience, etc. However, the answers
are only ``yes'' in a trivial, vacuous sense; the antecedent premise on which these questions are founded is vehemently
denied by this author. Activism is not, and has never been, the point of hacktivism. The cloak of Anonymous, in all of
its aspects: membership, activities and misinformation is merely a poorly executed smoke-and-mirror show, whose success
has been facilitated only by the public's imagination.

\end{document}
