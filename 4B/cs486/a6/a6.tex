\documentclass[12pt]{article}

\usepackage[margin=1in]{geometry}
\usepackage{amsmath, amssymb}
\usepackage{amsfonts}
\usepackage{braket, units, enumitem}
\usepackage{todonotes}

\let\oldBC\because
\renewcommand\because{\raisebox{0.75pt}{$\quad\oldBC\quad$}}

\begin{document}

\newcounter{set}
\setcounter{set}{1}
\newcounter{problem}[set]
\newcommand{\problem}{{\vspace{2\baselineskip}\noindent\large \bfseries Problem~\arabic{set}:}\\\refstepcounter{set}}
\newcommand{\problemsub}{\refstepcounter{problem}{\vspace{2\baselineskip}\noindent\large \bfseries \alph{problem}:}\\}
\newcommand{\problemasub}{\refstepcounter{problem}{\vspace{2\baselineskip}\noindent\large \bfseries Problem~\arabic{set}\alph{problem}:}\\}

\title{CS 486 - A6 \\ The Grand Challenge of Computer Go}

\author{Alexander Maguire \\
amaguire@uwaterloo.ca \\
20396195}

\setlength{\parindent}{0pt}
\maketitle


\problemsub % what is the motivation

The motivation for this paper, The Grand Challenge of Computer Go, is twinfold: ostensibly it exists as an introduction
to state-of-the-art methods for computers playing Go by providing algorithms which perform significantly better than
naive approaches. However, additionally, Gelly et al. seem to be interested in Go artificial intelligence qua artificial
intelligence itself; while they seem to be genuinely interested in improving the state-of-the-art of Go, they also
soliloquize at surprisingly length on ``classic two-player games [as] excellent test beds for AI.''



\problemsub % what is the proposed solution

Gelly et al. identify the major stumbling blocks of traditional AI approaches to Go to be the huge branching factor of
the game tree; that naive approaches would work \textit{in principle}, but the exponential growth of the tree makes it
infeasible to do. Additionally, they claim that it is hard to make a utility function for go, and instead a policy-based
approach is used. The solution they propose, is Monte-Carlo tree search (MCTS), which randomizes over game nodes and
in the limit approximates the optimal solution. An exponentially difficult problem is thus reduced to parameter tuning
of the algorithm in the expected case.



\problemsub % what is the evaluation of b)

The authors claim that this MCTS approach to Go makes significant improvement over previous solutions; older Go AIs were
incapable of beating good amateur players, but contemporary AI approaches have now beaten professional players (albeit
with a significant stone handicap).



\problemsub % what are the contributions

The Grand Challenge of Computer Go makes several contributions to the field of Go-related artificial intelligence. It
introduces the aforementioned MCTS, as well as some extensions: RAVE, essentially a dynamic programming approach to
subtree valuation; including both human priors as well as machine learned information to state valuation; and noting
that the MCTS approach lends itself well to parallelization, unlike more naive approaches like minimax.



\problemsub % what are the future directions

It is explicitly stated in the paper that MCTS approaches have proved useful in other domains, such as the games of
Amazons, Lines of Action, Hex, and Havannah (none of which were known \textit{a priori} to this author), and it is
likely that given such a wide applicability MCTS will find applications in more games which have yet found themselves
difficult to adapt to AI. Furthermore, while top-level chess AIs can reliably beat the best humans, this is not the case
yet for Go. Further improvements to MCTS seem promising on this front as well.

\end{document}
